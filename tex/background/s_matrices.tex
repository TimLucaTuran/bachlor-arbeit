\paragraph{Jones Formalism}
 A planar lightwave propagating along the z axis trough a homogeneous material can be described as:

\begin{equation}
    \vb E =
    \begin{pmatrix}
        E_x e^{i(kz - \omega t + \varphi_x)}\\
        E_y e^{i(kz - \omega t + \varphi_y)}\\
        0\\
    \end{pmatrix}
    =
    \qty(E_x \, e^{i \varphi_x} \, \va{e_x} +
         E_y \, e^{i \varphi_y} \, \va{e_y})
        e^{i(kz - \omega t)}
\end{equation}
\\

\noindent
the waves polarization is determined by the scaling factors of $\va{e_x}$ and $\va{e_y}$ and can be expressed as a \textit{Jones Vector} $\vb j \in \mathbb{C}^2$.

\begin{equation}
    \vb j = \frac{1}{\sqrt{E_x^2 + E_y^2}}
    \begin{pmatrix}
        E_x\\
        E_y \, e^{i \delta}
    \end{pmatrix}
    \qq{with}
    \delta := \varphi_y - \varphi_x
\end{equation}

\noindent
Now all linear operations on the polarization are matrices $\hat{M} \in \mathbb{C}^{2 \times 2}$. That means all passive components have a corresponding matrix. A couple examples for components in horizontal position:


\begin{equation}
\begin{split}
    \qq*{polarizer:} \hat{M} &=
    \begin{pmatrix}
        1 & 0\\
        0 & 0
    \end{pmatrix}\\
    \qq*{$\lambda / 4$ plate:} \hat{M} &=
    \begin{pmatrix}
        1 & 0\\
        0 & i
    \end{pmatrix}
    e^{-\frac{i \pi}{4}}\\
    \qq*{$\lambda / 2$ plate:} \hat{M} &=
    \begin{pmatrix}
        -i & 0\\
        0 & i
    \end{pmatrix}\\
\end{split}
\end{equation}

\noindent
By using matrices one can easily compute the effect of multiple components using the matrix product and find the behavior of a rotated component $\hat{M}_{\varphi}$ using the standard rotation matrix.

\begin{equation}
    \hat{M}_{\varphi} = \hat{R}(-\varphi) \, \hat{M} \, \hat{R}(\varphi)
    \qq{where}
    \hat{R}(\varphi) =
    \begin{pmatrix}
        \cos \varphi & \sin \varphi \\
        -\sin \varphi & \cos \varphi
    \end{pmatrix}
\end{equation}
\paragraph{SASA}

\begin{equation}
    \hat{S} =
    \begin{pmatrix}
        \hat{T}^f & \hat{R}^b \\
        \hat{R}^f & \hat{T}^b
    \end{pmatrix}
\end{equation}
\note{How do symmetries in the MS result in $S$-Matrix symmetries.}

\paragraph{?Applying Jones calculus to MS?}
\note{Here: Under which conditions can one describe the effect of a MS with a jones matrix?}


\paragraph{Boundary Conditions}

\note{Maybe mark equations used as boundary conditions for the algorithm differently(e.g. fat)}
