To implement the desired algorithm we need to be able to calculate the optical behavior of stacked Meta Surfaces. The mathematical framework we will use is called Scatter Matrix Calculus and this section will give some insight into its physical origin and how to use it. We will start at the very beginning with the Maxwell Equations in matter.
\newlparagraph{Maxwell Equations}

\\

\noindent
\begin{tabular*}{\textwidth}{ll}
\begin{minipeqn}
    \curl{\vb{E}(\vb{r}, t)} = - \pdv{t} \vb{B}(\vb{r}, t)
\end{minipeqn}&
\begin{minipeqn}[c]
    \div \vb{D}(\vb{r}, t) = \rho_\s{ext}(\vb r, t)
\end{minipeqn}\\
\begin{minipeqn}
    \curl \vb H(\vb r, t) = \vb j(\vb r, t) + \pdv{t} \vb D(\vb r, t)
\end{minipeqn}&
\begin{minipeqn}[c]
    \div \vb B(\vb r, t) = 0
\end{minipeqn}
\end{tabular*}
\\
\\


\noindent
The four involved fields are:
$\vb E$...electric field, $\vb B$...magnetic flux density, $\vb D$...electric flux density and $\vb H$...magnetic field and the sources are the external charges $\rho_\s{ext}$ and the macroscopic currents $j$. All the material properties are captured by the $\vb D$ and $\vb H$ fields which are defined as follows:
\\

\begin{equation}\label{eq:bg:D}
\begin{aligned}
    \vb D(\vb r, t) &= \varepsilon_0 \vb E (\vb r, t) + \vb P (\vb r, t)\\
    \vb H(\vb r, t) &= \frac{1}{\mu_0} \qty[\vb B(\vb r, t) - \vb M(\vb r, t)]
\end{aligned}
\end{equation}

\\
\\
Where $\vb P$ is the dielectric polarization and $\vb M$ is the magnetic polarisation. One can read equation \ref{eq:bg:D} in the following way:
When the electric field $\vb E$ interacts with matter it exerts a force on all its charges and displaces them by a small amount. The separation of charges results in a counter field $\vb P$ and the total field $\vb D$ is now a superposition of $\vb E$ and $\vb P$.
This set of equations describes the whole electromagnetic spectrum, in this work however we are only interested in visible (VIS) and near infra red (NIR) light so we can make some simplifications. $\rho_\s{ext}$ = 0 and $\vb M = 0$ \note{why?}. Inserting these assumptions into the maxwell equation gives:
\\

\noindent
\begin{tabular*}{\textwidth}{ll}
\begin{minipeqn}\label{eq:bg:M1}
    \curl{\vb{E}(\vb{r}, t)} = - \mu_0 \pdv t \vb H(\vb r, t)
\end{minipeqn}&
\begin{minipeqn}[c]\label{eq:bg:M2}
    \varepsilon_0 \div \vb{E}(\vb{r}, t) = - \div \vb P(\vb r, t)
\end{minipeqn}\\
\begin{minipeqn}\label{eq:bg:M3}
    \curl \vb H(\vb r, t) = \vb j(\vb r, t) + \pdv{t} \vb P(\vb r, t)
    + \varepsilon_0 \pdv{t} \vb E(\vb r, t)
\end{minipeqn}&
\begin{minipeqn}[c]
    \div \vb H(\vb r, t) = 0
\end{minipeqn}
\end{tabular*}


\newlparagraph{Light in Vacuum}
Now we can derive the famous wave equation by considering $\curl$ \eqref{eq:bg:M1}:


\begin{equation}
\begin{aligned}
    \curl \bigg [\curl \vb E \bigg ] &= \curl[- \mu_0 \pdv t \vb H]\\
    \Leftrightarrow
    \grad(\div \vb E) - \Delta \vb E &=
    - \mu_0 \pdv t \curl \vb H
    \qquad \bigg | \qq{subs. \eqref{eq:bg:M3} and \eqref{eq:bg:M2}}
    \\
    \Leftrightarrow
    \frac{1}{c^2} \pdv[2] t \vb E - \Delta \vb E
    &= -\mu_0 \pdv t \vb j - \mu_0 \pdv[2] t \vb P +
    \frac{1}{\varepsilon_0} \grad(\div \vb P)
\end{aligned}
\end{equation}

In vacuum ($\vb P = 0$ and $\vb j = 0$) the right side of this equation vanishes and we are left with\\
$\frac{1}{c^2} \pdv[2] t \vb E - \Delta \vb E = 0$ which is solved by the plane wave $\vb E = \vb E_0 e^{i(\vb k \vb r - \omega t)}$ where $\frac{\omega}{k} = c$. This describes the propagation of light through empty space: a sinusoidal occilation in time and space along the $\vb k$ direction where $\vb E, \, \vb B$ and $\vb k$ are all perpendicular to each other.
\note{show figure?}
\clearpage

\newlparagraph{Light in homogeneous, isotropic materials}
\noindent
The next question we can answer is how light propagates through a homogeneous and isotropic material. For us the dielectric polarization is some linear function of the electric field so $\vb P(\vb r, t) = \hat{\chi}(\omega, \vb r) \vb E(\vb r, t)$. An isotropic material behaves the same for all orientations of $\vb E$ that means $\hat{\chi}(\omega, \vb r)$ becomes a scalar property $\chi(\omega, \vb r)$. If the material is additionally homogeneous, that is the same everywhere independent of $\vb r$, then $\div \chi(\omega, \vb r) = 0$. With equation \eqref{eq:bg:M2} this gives us $\div \vb P = 0$
and the wave equation simplifies to:

\begin{equation}
\begin{aligned}
    $&\frac{\varepsilon}{c^2} \pdv[2] t \vb E - \Delta \vb E = 0$\\
    \qq*{where} &\varepsilon \, \vb E := \qty(1 + \chi) \vb E = \vb E + \vb P
\end{aligned}
\end{equation}

So light behaves in these materials exactly as it would in vacuum we only have to account for a decreased speed of light
$c' = \frac{c}{\sqrt{\varepsilon}} =: \frac{c}{n}$
with the refractive index $n$.
A complex valued $n := \eta + i \kappa$ even captures the possibility of a decaying field:

\begin{equation}
    \vb E = \vb E_0 e^{i(n \vb k \vb r - \omega t)}
    = \vb E_0 \
    \underbrace{e^{-\kappa \vb k \vb r}}_\s{
    decay} \
    \underbrace{e^{i(\eta \vb k \vb r - \omega t)}}_\s{
    oscillation}
\end{equation}

\newlparagraph{Fresnel Equations}
The meta surface stacks we want to use are obviously not one homogeneous material. They contain many interfaces between different materials and we can again use the maxwell equations to predict how light will behave at such an interface. 
