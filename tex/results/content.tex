\section{Conclusion} \label{sec:conclusion}

\subsection{Summary}
We started off by discussing the physics behind the chosen optical system of stacked plasmonic metasurfaces. 
We learned how Maxwell's equations govern the behavior of light at the interfaces between layers and how to use the $S$-matrix calculus and SASA to describe stacks as a whole. This calculus came with a number of boundary conditions the algorithm needed to obey. Then we turned to neural networks and discussed how they are build and trained. We settled on Convolutional Neural Networks as they would be able to efficiently use the spatial information in a transmission spectrum.
\\

\indent
The first step in implementing the algorithm was training a network on spectra generated by SASA. Here, for the first time, the many-to-one problem came up where multiple designs mapped to a single spectrum. We had already derived that, under certain conditions, metasurface stacks would produce the same spectrum in the top-to-bottom and bottom-to-top orientations. After removing these equivalent stacks the network could be trained successfully. It was very accurate in finding the discrete design parameters to a spectrum but the predictions for the continuous parameters were only good enough as an initial guess.
This is why the another optimization step was added. A conventional simplex further tunes the continuous parameters and at every step the performance of the new stack is evaluated via SASA. This optimization loop is repeated until the target accuracy is reached.


\subsection{Evaluation}
The algorithm is generally able to reproduce spectra from stacks generated based on the database. This is a good first test as we know these spectra are possible and if the algorithm could not reproduce them something would be wrong. However, the goal was not to reproduce known spectra but being able to find design parameters to new ones. Here the evaluation has to be more nuanced. Yes, some new transmission spectra like Gauß peaks, dips and sinus functions are possible but the first limitation becomes obvious. The maximum transmission has to stay below $\sim 50\%$. This can be understood by the choice in materials. Plasmonic metasurfaces will always dissipate energy from the field via ohmic losses. Another limitation can be seen in figure . Even when the target function is approximated successfully there might be additional unwanted features in the spectrum. 
\\

\indent
Almost comically, some targets that would be trivial for a human are very challenging for the algorithm. If a human was tasked to create the maximum possible transmission at all wavelengths he or she would probably use zero thickness metasurfaces or metasurfaces which only consist of holes. When the algorithm is task with a transmission spectrum that is one everywhere it gets confused to a point of predicting negative parameters even though these are completely out of bounds and very negatively affect the simplexes loss function through equation \eqref{eq:al:simlex_loss}. However, even this behavior can be explained because the are no zero thickness or hole only metasurfaces in the database we cannot expect the simplex to know about them. Additionally, the less reachable a task is, the higher is the total loss and the less important is the boundary term explaining the nonsensical negative parameters. This leads to the next section of what could possibly be improved.
\subsection{Outlook}