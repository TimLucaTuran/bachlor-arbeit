\section{Conclusion} \label{sec:conclusion}

\subsection{Summary}
We started off by discussing the physics behind the chosen optical system of stacked plasmonic metasurfaces. We learned how Maxwell's equations govern the behavior of light at the interfaces between layers and how to use the $S$-matrix calculus and SASA to describe stacks as a whole. This calculus came with a number of boundary conditions the algorithm needed to obey. Then we turned to neural networks, how they are build and trained. We settled on Convolutional Neural Networks as they would be able to efficiently use the spatial information in a transmission spectrum.
\\

\indent
The first step in implementing the algorithm was training a network on spectra generated by SASA. Here, for the first time, the many-to-one problem came up where multiple designs maped to a single spectrum. We had already derived that, under certain conditions, metasurface stacks would produce the same spectrum in the top-to-bottom and bottom-to-top orientations. After removing these equivalent stacks the network could be trained successfully. It was very accurate in finding the discrete design parameters to a spectrum but the predictions for the continuous parameters were only good enough as an initial guess.
\subsection{Evaluation}

\subsection{Outlook}