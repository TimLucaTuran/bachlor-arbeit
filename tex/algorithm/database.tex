\begin{tabular}{ll}
    \toprule
    Input: & layer parameters $p = (w, \, l, \, t, \, \Lambda)$, material and geometry \\
    Output: & approximation of the $S$-matrix of this layer $\hat{S} = \hat{S}(p)$\\
    \bottomrule
\end{tabular}
\\
\\
The database consist of $\approx$ 5000 $S$-matrices of single layers which were simulated with the Fourier Modal Method (FMM) on a compute cluster. Its main input are the layer parameters width, length, thickness and period, so $p = (w, \, l, \, t, \, \Lambda)$. This module first looks for the $n$ closest neighbors of $p$. To do that the input is scaled:

\begin{equation}
    \tilde{p}_i = \frac{p_i - p_i^\s{min}}{p_i^\s{max} - p_i^\s{min}}
    \qq{so that} \tilde{p}_i \in [0, \, 1]
\end{equation}

\noindent
Now the distance $d$ to every entry in the database satisfying the material geometry combination is calculated and the $n$ entries with the smallest distance are selected:

\begin{equation}
    d(p, \,q) = \sum_i \ \qty|p_i - q_i|
\end{equation}

\noindent
The output $\hat{S}(p)$ is calculated via Inverse Distance Weighting \cite{Shepard1968} so that more distant entries have a smaller effect. Let
$(q_1, \, ... \, , \, q_n)$ be the $n$ closest neighbors to $p$ with the stored $S$-matrices
$(\hat{S}_1, \, ... \, , \, \hat{S}_n)$
then:

\begin{equation}
\begin{aligned}
    \hat{S}(p) = \sum_{j=1}^n \ w_j \, \hat{S}_j
    \qq{where}& w_j = \frac{1 / d_j^2}{\sum_i \ 1 / d^2_i} \\
    \qq*{so that} &\sum_j \ w_j = 1
\end{aligned}
\end{equation}

\noindent
To have this interpolated approximation $\hat{S}(p)$ be close to the result of rigoros simulation FMM($p$) the simulated grid of parameters needs to be sufficiently dense. \note{Insert some calculation of what is sufficiently dense}
