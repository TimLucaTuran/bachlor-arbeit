% !TEX root = ../bachor-arbeit.tex
\begin{tabular}{ll}
    \toprule
    Input: & Layer parameters 
    $\mc L = (w, \, l, \, t, \, \Lambda, \, m, \, g)$\\
    Output: & Interpolation of the $S$-matrix for this layer \
    $\hat{S} = \hat{S}(\mc L)$\\
    \bottomrule
\end{tabular}
\\

\paragraph{Fourier Modal Method} ~\\
\label{sec:FMM}The database consist of $\sim 5000$ $S$-matrices of single layers which were simulated with the Fourier Modal Method (FMM) on a compute cluster. This method is applicable to all surface structures which are periodic in $x$ and $y$ direction and constant in $z$ direction. It works by expanding the involved fields into their diffraction orders via a Fourier expansion. For example, for the reflected electric field is

\begin{equation} \label{eq:al:fmm}
    \vb E_\s r = \sum_{m,n} \vb R_{mn} e^{i \vb k_{mn} \vb r}.
\end{equation}

Applying Maxwell's equations and the continuity conditions described in section \ref{sec:s_mats} results in an eigenvalue problem which can be reformulated into a linear set of equations. The unknown properties like $R_{mn}$ are found by solving this set of equations. Computationally, a Fourier series like \eqref{eq:al:fmm} has to be truncated at some order. This order determines the accuracy of the FMM and the matrix which represents the set of linear equations is sized order $\times$ order. Because of this the computational complexity of this method increases rapidly with the order. The method was first introduced by Noponen and Turunen \cite{Noponen1994}.

\paragraph{Interpolation} ~\\
To find the $S$-matrix to layer parameters $\mc L$, that are not already stored in the database, this module has to interpolate between the pre simulated $S$-matrices. First, it looks for the $n$ closest neighbors of $\mc L$. To do this, the continuous input is normalized through

\begin{equation}
    \bar{\mc L}_i = \frac{\mc L_i - \mc L_i^\s{min}}{\mc L_i^\s{max} - \mc L_i^\s{min}}
    , \  i \in 1 ... 4
    \qq{so that} \bar{\mc L}_i \in [0, \, 1].
\end{equation}

\noindent
Now the distance $d$ to every entry in the database satisfying the material geometry combination is calculated and the $n$ entries with the smallest distance are selected where

\begin{equation}
    d(\mc L^1, \, \mc L^2) := \sum_{i=1}^4 \ \qty|\mc L^1_i - \mc L^2_i|
\end{equation}

\noindent
The interpolation $\hat{S}(\mc L)$ is calculated via Inverse Distance Weighting \cite{Shepard1968} so that more distant entries have a smaller effect on the result. Let
$(\mc L^1, \, ... \, , \, \mc L^n)$ be the $n$ closest neighbors to $\mc L$ with stored $S$-matrices
$(\hat{S}_1, \, ... \, , \, \hat{S}_n)$
and
$d_j = d\qty(\mc L ,\, \mc L^j)$
then

\begin{equation}
\begin{aligned}
    \hat{S}(\mc L) = \sum_{j=1}^n \ w_j \, \hat{S}_j,
    \qq{where}&
    w_j = \frac{1 / d_j^2}{\sum_i \ 1 / d^2_i} \\
    \qq*{so that} 
    &\sum_j \ w_j = 1.
\end{aligned}
\end{equation}

The exact parameters of the simulated $S$-matrices can be found at
\url{https://github.com/TimLucaTuran/sasa_db/blob/master/data/NN_smats.xlsx}.