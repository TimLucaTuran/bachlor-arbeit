% !TEX root = ../bachlor-arbeit.tex
\begin{tabular}{ll}
    \toprule
    Input: &
    \begin{tabular}[t]{@{}l@{}l@{}}
        current spectrum $I$ a $\qty|\vb{\lambda}| \times 2$ Array, \\
        current continuous parameters
        $p = (w,l,t,\Lambda,h,\varphi)$\\
        $h$...spacer height, $\varphi$...layer rotation
    \end{tabular}\\
    Output: & improved continuous parameters $\tilde p$\\
    \bottomrule
\end{tabular}
\\
\\
The optimizer is at the core a Downhill-Simplex \cite{Nelder1965} minimizing the mean-squared-difference between the current spectrum $I_\s{c}$ and target spectrum $I_\s{t}$ so it minimizes
$C_\s{mse}\qty(I_\s{c}(p), \, I_\s{t})$
This standard method is however unable to follow the constraints and has to be modified in that regard. To achieve this one can introduce a distance to the boundary $D$. Let $p$ be a single parameter with lower bound $p^l$ and upper bound $p^u$, then:

\begin{equation}
    D\qty(p, \, p^l, \, p^u) =
    \begin{cases}
        p^l - p, & \text{for } p < p^l\\
        0, & \text{for } p^l \leq p \leq p^u\\
        p - p^u, & \text{for } p^u < p
    \end{cases}
\end{equation}

\noindent
In this way one can penalize the simplex for stepping over the set boundaries by using a total loss $L$ which depends on the sum of all distances $D_i$:

\begin{equation}
    L(I_\s{c}, \,I_\s{t}, \,p) =
    {\underbrace{%
    \vphantom{ \left(\frac{a^{0.3}}{b}\right) }
    C_\s{mse}(I_\s{c}, \, I_\s{t})}_{\text{find target}}}
    +
    {\underbrace{%
    \qty[\sum_i D\qty(p_i, \, p_i^l, \, p_i^u)]^2
    }_{\text{stay within bounds}}}\\
\end{equation}
\note{align underbraces}
\\
\\
\noindent
The choice of power depends on how much the simplex should be penalized for stepping over a boundary. All our conditions are requirements for physical approximations and
\\
\note{Maybe 3D plot of an example loss function}
