% !TEX root = ../bachlor-arbeit.tex
Metamaterials are materials with sub-wavelength structure whose physical properties are determined by their internal geometry. They can be engineered to exhibit properties not found in nature like negative refractive indices \cite{Shelby2001}.
Usually metamaterials are organized into periodic unit cells called metaatoms and a metasurface is just a single 2D layer of these metaatoms.
Normal optical devices like lenses or phase plates rely on distances much larger than one wavelength to change the light's properties in contrast metasurfaces with their sub-wavelength thickness enable thin components not possible before \cite{Yu2014}.
These metasurfaces promise custom optical components which can be tailored to a wide range of applications just by changing the surface geometry
but because they need to have sub-wavelength structure optical metasurfaces with complex geometries are quite hard to manufacture. Also predicting the optical behavior of metasurfaces cannot be done analytically and involves computationally intensive simulations which slows the design process. 
\\

\indent
2016 Menzel and Sperrhake \cite{Menzel2016} presented an algorithm called SASA to analytically calculate the optical behavior of metasurface stacks if one knows how the surfaces behave individually. This allows for a different approach when designing an optical component. The idea here is to create a target behavior not by using complex geometry but by using simple metasurfaces and stacking them on top of each other. Designing a component becomes an optimization problem of choosing the right surfaces and then tuning stack parameters like the distance between layers and their rotation angle. This approach is possible because SASA is analytical in nature and many different stacks can be considered very quickly but it also poses some new challenges when trying to automate the process. Conventional optimization methods rely on the continuity of the underlying process so that if a parameter is tuned slightly it is possible to calculate a gradient and tell wether that was a step in the right direction. This is a problem because some of the choices involved when building a stack are categorical. For example, choosing the material of the layers or the kind of geometry.
\\

\indent
Another recent development has produced a very general and powerful way of dealing with categorical decision problems. Deep Neural Networks have found their way into virtually all areas of life from autonomous driving \cite{Dequaire2016} to video recommendations \cite{youtube}. Given enough data it should be possible to train a Neural Network to choose the discrete stack parameters to a given optical target. Again SASA being analytical is useful because it can generate a lot of training stacks very quickly. This sets the goal for this work: Write an algorithm which can find a metasurface stack to a custom optical target and do so by training a Neural Network on data generated by the SASA algorithm.
\\

\indent
In section \ref{sec:background} we will first develop the physical theory to understand metasurface stacks and then look into the computer science background of Neural Networks to decide which network architecture fits the problem at hand. The Algorithm is described in detail in section \ref{sec:algorithm}, improved upon in section \ref{sec:inverse} and evaluated in section \ref{sec:results}.
All the code written for this project is open source and documented thoroughly at \url{https://github.com/TimLucaTuran}.
