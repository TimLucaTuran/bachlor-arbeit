% !TEX root = ../bachlor-arbeit.tex
\begingroup
\renewcommand{\arraystretch}{1.5}
\begin{tabular}{m{4cm} l}
Symbols & Explanation \\
\midrule
$\vb E, \, \vb B, \, \vb k$ & Vectors are written bold \\
$E, \, B, \, k $ &
Amplitudes of vectors are written non-bold so that $\qty|\vb E| = E$ \\
$\hat S, \, \hat J, \, \hat w$ & Matrices have an hat \\
$\hat w^2_{1,2}, \, E^\s{in}_x$ & Super and sub indices are used to specify matrix, vector or tuple elements \\
$\qty(y_i - y_i')^2, \, (n)^2$ & All exponents are outside parentheses to differentiate between index and exponent \\
$\mathds 1$ & Unity matrix sized accordingly e.g. in the context of $2 \times 2$ matrices
$\mathds 1 = 
\begin{pmatrix}
    1 & 0\\
    0 & 1
\end{pmatrix}$\\
$\nabla$ & Nabla operator
$\nabla
= \begin{pmatrix}
    \pdv x \\
    \pdv y \\
    \pdv z
\end{pmatrix}$ \\
$\Delta$ & Laplace operator $\Delta =
\begin{pmatrix}
    \pdv[2] x + \pdv[2] y + \pdv[2] z \\
    \pdv[2] x + \pdv[2] y + \pdv[2] z \\
    \pdv[2] x + \pdv[2] y + \pdv[2] z 
\end{pmatrix}$\\
$\odot$ & Hadamard product
    $\vb a \odot \vb b :=
    \begin{pmatrix}
        a_1 \, b_1 \\
        a_2 \, b_2 \\
        \vdots
    \end{pmatrix}$\\
\end{tabular}
\endgroup
